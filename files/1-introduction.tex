\section{Introduction}

% General memory
Memory is the ability to store and retrieve information.
We can distinguish procedural memory and declarative memory.
Procedural memory is a type of memory that does not require conscious recall and is mostly related to motor tasks, while declarative memory is the ability of a conscious recall of information.
Declarative memory can be itself divided into subcategories: semantic and episodic memory.
Episodic memory stores past experiences and their emotional associations, while semantic memory stores and recalls facts independent of the context \parencite{squire2009}.

Memory, and especially semantic memory, can be tested in several ways.
In an associative learning task, two stimuli are mapped together e.g. two words.
The subject is then presented with one element of the stimuli pair and has to recall the corresponding other --- that is, to recall from a partial cue.
Another test to assess memory is with free recall tasks.
In this type of task, a subject is presented with a set of items to memorize.
Later, the subject is asked to recall as many items as possible \parencite{tulving2000}.

% Concerning semantic memory,
Previous literature has shown that recalling memory items in the absence of cues is a difficult task: subjects usually fail to recall more than short lists of items in a free recall task \parencite{murdock1960}. %
% Reintroduction of associative learning: subjects create by themselves connections in free recall conditions
However, according to the \textit{Search of Associative Memory} (SAM) model, associations between memory items influence memory recall even in the absence of partial cues.
From a neuroscience perspective, this could be explained by the overlaps between neuronal representations of memories \parencite{raaijmakers1981, romani2013}.

% Aims, goals original article
Recanatesi \textit{et al.} \parencite{recanatesi2015} present a model of memory retrieval based on a Hopfield model for associative learning, with network dynamics that reflect associations between items due to semantic similarities.

% Technical details
Indeed, transitions occur due to the activation of populations of neurons encoding for a memory item.
This sequential activation of neuronal ensembles forms stable states at different domain regions of a periodic function, which provides inhibition to the network.
Network dynamics are also compatible with empirical observations about free recall previously described \parencite{recanatesi2015}.

% Replication work
In the present work, we proceed to replicate the model as presented by Recanatesi \textit{et al.} \cite{recanatesi2015}.
During our replication efforts, we discover several errors in parameters and collaborate with the original work authors to provide a successful replication and correct the original article.